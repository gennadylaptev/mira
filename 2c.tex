\documentclass[letterpaper, 12pt, english]{article}
\usepackage[letterpaper, total={10in, 8in}, margin=1.5in]{geometry}
\usepackage{parskip}  % turn off paragraph indentation

\usepackage[utf8]{inputenc}
\usepackage[english]{babel}
\usepackage{tikz-cd}
\usetikzlibrary{babel}
\selectlanguage{english}

\usepackage{amsmath}  % for the equation* environment
\usepackage{amsthm}   % for definition env
\usepackage{xcolor}   % textcolor
\usepackage{amsfonts}
\usepackage{amssymb}
\usepackage{mathtools} % fancy arrows
\usepackage{graphicx}  % images
\usepackage{hyperref}  % hyperlinks

% numbered envs for definition, lemma, proposition, remark, theorem
\theoremstyle{definition}
\newtheorem{defn}{DEFINITION}[section]

\theoremstyle{definition}
\newtheorem{problem}{Exercise}

\theoremstyle{definition}
\newtheorem{example}{Example}[section]

% solution env
\newenvironment{sol}{\begin{proof}[Solution]}{\end{proof}}

\theoremstyle{plain} % default
\newtheorem{lemma}{Lemma}[section]

\theoremstyle{plain} % default
\newtheorem{prop}{Proposition}[section]

\theoremstyle{plain} % default
\newtheorem{cor}{Corollary}[section]

\theoremstyle{plain} % default
\newtheorem{thrm}{Theorem}[section]

\theoremstyle{remark}
\newtheorem{remark}{Remark}[section]

% empty set symbol
\let\oldemptyset\emptyset

\title{2C}
\author{Gennady Laptev}
\date{}

\begin{document}
\maketitle
\setcounter{section}{0}


% 1
\setcounter{problem}{0}
\begin{problem}
Explain why there does not exist a measure space $ (X, S, \mu) $ with the property
$ \left\{ \mu(E) \,\middle|\, E \in S \right\} = [0,1) $.
\begin{sol}
    Suppose it is true. Let's denote $ M =  \left\{ \mu(E) \,\middle|\, E \in S \right\}$.

    Since $ S $ is a $ \sigma $-algebra on $ X $, then $ X \in S $. 
    Suppose $ \mu(X) = x $ and, by hypothesis, $ x < 1 $.
    Since measures preserve order, then for any $ A \in S $ 
    holds $ \mu(A) \leq \mu(X) $.
    Assume $ x > 0 $, otherwise 
    $ M = \{ 0 \} $.
    Then $  x = \sup M < 1$.
    Therefore, for any $ \varepsilon > 0 $ there exists $ y \in M $ 
    s.t. $ y + \varepsilon > x $. Taking $ \varepsilon = \frac{1 - x}{2} $,
    we get
    \begin{gather*}
        y + \varepsilon = y + \frac{1 - x}{2} \leq x + \frac{1 - x}{2} 
        < x + 1 - x = 1.
    \end{gather*}
    So there exist $ z = y + \varepsilon $ s.t. $ z \not \in M $ and $ z \in [0, 1) $.
    Thus, $ M \ne [0,1) $.
\end{sol}
\end{problem}


% 4
\setcounter{problem}{3}
\begin{problem}
Give an example of a measure space $ (X, S, \mu) $ such that
\begin{gather*}
    \left\{ \mu(E) \,\middle|\, E \in S \right\} =
    \{ \infty \} \cup \bigcup_{k=1}^{ \infty} [3k, 3k +1]
\end{gather*}
\begin{sol}
%    Consider $ X = \mathbb{Z}^{+} \times \{ 0, 3, \ldots \} $ and 
%    $ S $ as a set of all subsets of $ X $ s.t. every $ A \in S $
%    has the form $ B \times  C $, where $ B \subset \mathbb{Z}^{+} $,
%    $ C \subset \{ 0, 3, \ldots \} $.
%    Let $ p_1: X \rightarrow \mathbb{Z}^{+}$, 
%    $ p_2: X \rightarrow  \{ 0, 3, \ldots \}$ be natural projections.
%    Define a measure $ \mu $ on $ (X, S) $ for each 
%    $ E \subset X $ as follows:
%    \begin{gather*}
%        \mu(E) = \sum_{x \in p_1(E)} x + \sum_{y \in p_2(E)} 2^{-y}.
%    \end{gather*}
%    Clearly, $ \mu( \emptyset) = 0 $.
%    For a list of disjoint $ E_1, E_2, \ldots $ we have
%    \begin{gather*}
%        p_{j} \left( \cup_{k=1}^{ \infty} E_k \right)  =
%        \cup_{k=1}^{ \infty} p_{j} (E_k)
%    \end{gather*}
%    
%    \begin{gather*}
%        \mu (\cup_{k=1}^{ \infty} E_k) =
%        \mu \left(  
%        \sum_{x \in p_1(\cup_{k=1}^{ \infty} E_k)} x +
%        \sum_{y \in p_2(\cup_{k=1}^{ \infty} E_k)} 2^{-y}
%        \right) = \\ 
%        \mu \left(  
%        \sum_{x \in \cup_{k=1}^{ \infty} p_{1} (E_k)} x +
%        \sum_{y \in \cup_{k=1}^{ \infty} p_{2} (E_k)} 2^{-y}
%        \right) =
%    \end{gather*}
%    
\end{sol}
\end{problem}

% 6
\setcounter{problem}{5}
\begin{problem}
Find all $ c \in [3, \infty) $ s.t. there exists a measure space $ (X, S, \mu) $ with
\begin{gather*}
    \left\{ \mu(E) \,\middle|\, E \in S \right\} = [0,1] \cup [3, c]
\end{gather*}
\begin{sol}
    Since measures preserve order, then we have $ \mu(X) = c $. 
    Suppose $ c < 4 $. Then by hypothesis there exists $ E \subset X $ 
    s.t $ \mu(E) = 1 $, therefore $ \mu(X - E) = \mu(X) - \mu(E) = c - 1 < 3 $,
    contradiction. So, $ c \geq 4 $.

    Assume $ c= 4 $. Then 
    there exists $ E \subset X $ s.t $ \mu(E) = 3 + a, a \in [\frac{1}{2}, 1) $,
    and there exists $ F \subset X $ s.t. $ \mu(F) = b > a, b \in (\frac{1}{2}, 1] $.
    If $ F \subset E $, then $ \mu(E - F) = 3 + a - b < 3 $, contradiction,
    therefore, $ F \cap E = \emptyset $. Thus, $ \mu(E \cup F) = 3 + a+ b > 4 $, so
    $ c > 4 $.

    Now assume $ c = k \geq 5 \in \mathbb{N}$. Then there exist $ E, F \subset X $ s.t.
    $ \mu(E) = k - \varepsilon, \varepsilon \in (0,1) $, $ \mu(F) = k -2 $.
    Assume $ F \subset E $, then we get
    $ \mu(E - F) = k - \varepsilon - k + 2 = 2 - \varepsilon $, contradiction. So,
    $ E \cap F = \emptyset $, therefore
    $ \mu(E \cup F) = k - \varepsilon + k - 2 = 2k - 2 -\varepsilon > k + 1 $, since 
    $ k > 2 $. We get that $ c $ is greater than any natural number, and by Archimedean 
    property, greater than any real number. Thus, $ c = \infty $.
\end{sol}
\end{problem}

% 10
\setcounter{problem}{9}
\begin{problem}
Give an example of a measure space $ (X, S, \mu ) $ and a decreasing sequence
$ E_{1} \supseteq E_{2} \supseteq \ldots $ of sets in S such that 
\begin{gather*}
    \mu \left( \bigcap_{k=1}^{ \infty} E_{k} \right) \ne \lim_{k \to \infty} \mu(E_{k}).
\end{gather*}
\begin{sol}
   Consider $ \mathbb{R} $ with a counting measure $ \mu $ which is
   defined on each $ E \subset \mathbb{R} $ as
   \begin{gather*}
       \mu(E) = 
      \begin{cases}
          n, ~\textrm{if}~ E ~\textrm{is finite and has}~n ~\textrm{elements}, \\ 
          \infty, ~\textrm{otherwise}.
      \end{cases}
   \end{gather*}
   Choose $ E_k = (k, \infty)$.

   Clearly $ \bigcap_{k=1}^{ \infty} E_{k} = \emptyset $ (Suppose it is not true.
   Then there exists a real number $ x \in \bigcap_{k=1}^{ \infty} E_{k} $,
   that is, $ x \in (k, \infty) $ for every $ k \in \mathbb{N} $.
   By Archimedean property, there exists a natural $ n $ s.t. $ n > x $, 
   so $ x \not \in (n, \infty) $, which leads to contradiction). Thus, 
   $ \mu \left( \bigcap_{k=1}^{ \infty} E_{k} \right) = 0 $. 
   But $ \mu(E_k) = \infty $ for every $ k $, therefore
   $ \lim_{k \to \infty} \mu(E_{k}) = \infty $  which gives the desired inequality.
\end{sol}
\end{problem}
 
\end{document}
