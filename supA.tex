\documentclass[letterpaper, 11pt, english]{article}
\usepackage[letterpaper, total={10in, 8in}, margin=1in]{geometry}
\usepackage{parskip}  % turn off paragraph indentation

\usepackage[utf8]{inputenc}
\usepackage[english]{babel}
\usepackage{tikz-cd}
\usetikzlibrary{babel}
\selectlanguage{english}

\usepackage{amsmath}  % for the equation* environment
\usepackage{amsthm}   % for definition env
\usepackage{xcolor}   % textcolor
\usepackage{amsfonts}
\usepackage{amssymb}
\usepackage{mathtools} % fancy arrows
\usepackage{graphicx}  % images
\usepackage{hyperref}  % hyperlinks

% numbered envs for definition, lemma, proposition, remark, theorem
\theoremstyle{definition}
\newtheorem{defn}{DEFINITION}[section]

\theoremstyle{definition}
\newtheorem{problem}{Exercise}

\theoremstyle{definition}
\newtheorem{example}{Example}[section]

% solution env
\newenvironment{sol}{\begin{proof}[Solution]}{\end{proof}}

\theoremstyle{plain} % default
\newtheorem{lemma}{Lemma}[section]

\theoremstyle{plain} % default
\newtheorem{prop}{Proposition}[section]

\theoremstyle{plain} % default
\newtheorem{cor}{Corollary}[section]

\theoremstyle{plain} % default
\newtheorem{thrm}{Theorem}[section]

\theoremstyle{remark}
\newtheorem{remark}{Remark}[section]

% empty set symbol
\let\oldemptyset\emptyset

\title{Exercises A}
\author{Gennady Laptev}
\date{}

\begin{document}
\maketitle
\setcounter{section}{0}


% 1
\setcounter{problem}{0}
\begin{problem}
\begin{sol}
    \textbf{(a)} Let $ a, b, c \in \mathbb{F} $ where $ b, c $ are 
    additive inverses of $ a $. Then
    \begin{gather*}
        b = b + 0 = b + (a + c) = (a+b) + c = 0 + c = c.
    \end{gather*}
    That's why the cancellation property holds:
    \begin{gather*}
         x + y = x + z \implies -x + x +y = -x + x + z \implies y = z. 
    \end{gather*}
    
   \textbf{(b)} We have 
   \begin{gather*}
        a + (-a) = 0\\ 
        -(-a) + (-a) = 0.
   \end{gather*}
   Earlier we showed that the additive inverse is unique, therefore
   $ a = -(-a) $.

   \textbf{(c)} We have
   \begin{gather*}
       (-1)a + a = (-1)a + 1a = (-1 + 1)a = 0a = 0.
   \end{gather*}
   By \textbf{(a)} we get $ (-1)a = -a $.
\end{sol}
\end{problem}

% 2
\setcounter{problem}{1}
\begin{problem}
\begin{sol}
    \textbf{(a)} Let $ a, b, c \in \mathbb{F} $, where $ b,c $ are multiplicative inverses of 
    $ a $. Then we have
    \begin{gather*}
        b = 1b = (ac)b = (ab)c = 1c = c.
    \end{gather*}

   \textbf{(b)} We have
   \begin{gather*}
       a \cdot a^{-1} = 1 \\ 
       (a^{-1})^{-1} \cdot a^{-1} = 1.
   \end{gather*}
    By uniqueness of an inverse we obtain $ a =  (a^{-1})^{-1}$.   
\end{sol}
\end{problem}

% 3
\setcounter{problem}{2}
\begin{problem}
\begin{sol}
    First, note that $ (-1)(-1) = -(-1) = 1 $. Then we have
    $ (-a)(-b) = (-1)(-1)ab = ab$.
\end{sol}
\end{problem}



% 13
\setcounter{problem}{12}
\begin{problem}
\begin{sol}
    Suppose $ i \in \mathbb{F} $ s.t. $ i^{2} = -1 $. 
    Certainly, $ i \ne 0 $, since $ 0^2 = 0 \ne -1 $. Therefore, by properties of
    an ordered field, either $ i \in P $ or $ i \not \in P $.
    If $ i \in P $, then $ i^{2} \in P $, and we reach contradiction since $ -1 \not \in P $.
    Now suppose $ i \not \in P $ and
    consider $ i^2 \cdot i^2 = (-1) (-1) = 1 \in P $, and we again reach contradiction.
    Thus, such element doesn't exist.
\end{sol}
\end{problem}



 
\end{document}
