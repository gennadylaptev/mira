\documentclass[letterpaper, 11pt, english]{article}
\usepackage[letterpaper, total={10in, 8in}, margin=1in]{geometry}
\usepackage{parskip}  % turn off paragraph indentation

\usepackage[utf8]{inputenc}
\usepackage[english]{babel}
\usepackage{tikz-cd}
\usetikzlibrary{babel}
\selectlanguage{english}

\usepackage{amsmath}  % for the equation* environment
\usepackage{amsthm}   % for definition env
\usepackage{xcolor}   % textcolor
\usepackage{amsfonts}
\usepackage{amssymb}
\usepackage{mathtools} % fancy arrows
\usepackage{graphicx}  % images
\usepackage{hyperref}  % hyperlinks

% numbered envs for definition, lemma, proposition, remark, theorem
\theoremstyle{definition}
\newtheorem{defn}{DEFINITION}[section]

\theoremstyle{definition}
\newtheorem{problem}{Exercise}

\theoremstyle{definition}
\newtheorem{example}{Example}[section]

% solution env
\newenvironment{sol}{\begin{proof}[Solution]}{\end{proof}}

\theoremstyle{plain} % default
\newtheorem{lemma}{Lemma}[section]

\theoremstyle{plain} % default
\newtheorem{prop}{Proposition}[section]

\theoremstyle{plain} % default
\newtheorem{cor}{Corollary}[section]

\theoremstyle{plain} % default
\newtheorem{thrm}{Theorem}[section]

\theoremstyle{remark}
\newtheorem{remark}{Remark}[section]

% empty set symbol
\let\oldemptyset\emptyset

\title{1B}
\author{Gennady Laptev}
\date{}

\begin{document}
\maketitle
\setcounter{section}{0}

% 1
\setcounter{problem}{0}
\begin{problem}
\begin{sol}
    Clearly, $ L(f, P, [0,1]) = 0 $ for any partition $ P $,
    since any interval contains an irrational number, so
    $ \inf_{[a,b]} f = 0 $, where $ [a,b] \subset [0,1] $.

    Let $ q $ be a prime number. Then consider a partition
    $ P_q = [\frac{j-1}{q}, \frac{j}{q}]_{j = 0}^{q} $.
    We have $ \sup_{[\frac{j-1}{q}, \frac{j}{q}]} f = \frac{1}{q} $ 
    and $ U(f, P_q, [0,1]) = q \cdot \frac{1}{q^2} = \frac{1}{q}$,
    since we have $ q $ segments in partition of length $ \frac{1}{q} $.
    Then by Archimedean property for any $ \varepsilon > 0 $ we can find some
    $ n $ s.t. $ \frac{1}{n} < \varepsilon $, and we can always find a prime number $ q \geq n$, 
    since subset of prime numbers is not bounded in the set of natural numbers.
    Thus for every $ \varepsilon > 0 $ we can find a partition $ P_q $ s.t.
    \begin{gather*}
        U(f, P_q, [0,1]) - L(f, P_q, [0,1]) < \frac{1}{q}  < \varepsilon.
    \end{gather*}
    Then by problem 1A.3 we have that $ f $ is Riemann integrable and its integral is 0,
    since $ L(f,[0,1]) = 0 $.
\end{sol}
\end{problem}

% 2
\setcounter{problem}{1}
\begin{problem}
\begin{sol}
    $(\implies)$ We have
    \begin{gather*}
        L(-f, [a,b]) = \sup L(-f, P, [a,b]) = \sup -L(f, P, [a,b]) = \\ 
        \inf  L(f, P, [a,b]) = \inf (U(f, P, [a,b]) + \varepsilon) = 
        U(f, [a,b])
    \end{gather*}
    
\end{sol}
\end{problem}



\end{document}
