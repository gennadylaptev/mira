\documentclass[letterpaper, 11pt, english]{article}
\usepackage[letterpaper, total={10in, 8in}, margin=1.5in]{geometry}
\usepackage{parskip}  % turn off paragraph indentation

\usepackage[utf8]{inputenc}
\usepackage[english]{babel}
\usepackage{tikz-cd}
\usetikzlibrary{babel}
\selectlanguage{english}

\usepackage{amsmath}  % for the equation* environment
\usepackage{amsthm}   % for definition env
\usepackage{xcolor}   % textcolor
\usepackage{amsfonts}
\usepackage{amssymb}
\usepackage{mathtools} % fancy arrows
\usepackage{graphicx}  % images
\usepackage{hyperref}  % hyperlinks

% numbered envs for definition, lemma, proposition, remark, theorem
\theoremstyle{definition}
\newtheorem{defn}{DEFINITION}[section]

\theoremstyle{definition}
\newtheorem{problem}{Exercise}

\theoremstyle{definition}
\newtheorem{example}{Example}[section]

% solution env
\newenvironment{sol}{\begin{proof}[Solution]}{\end{proof}}

\theoremstyle{plain} % default
\newtheorem{lemma}{Lemma}[section]

\theoremstyle{plain} % default
\newtheorem{prop}{Proposition}[section]

\theoremstyle{plain} % default
\newtheorem{cor}{Corollary}[section]

\theoremstyle{plain} % default
\newtheorem{thrm}{Theorem}[section]

\theoremstyle{remark}
\newtheorem{remark}{Remark}[section]

% empty set symbol
\let\oldemptyset\emptyset

\title{2A}
\author{Gennady Laptev}
\date{}

\begin{document}
\maketitle
\setcounter{section}{0}

% 1
\setcounter{problem}{0}
\begin{problem}gg
\begin{sol}
    Consider $ I_1, \ldots $ such that
    $ B \subset \cup_{k=1}^{ \infty} I_k $.
    and $ J_k, \ldots $ s.t. $ A \subset 
    \cup_{k=1}^{ \infty} J_k$. We can merge both sets of intervals into 
    $ L_1, L_2, \ldots $
    with the following enumeration 
    \begin{gather*}
        L_i = 
       \begin{cases}
           J_{(i+1)/2}, & i~\textrm{is odd}, \\ 
           I_{i/2}, & i~\textrm{is even}.
       \end{cases}
    \end{gather*}
    Clearly, $ A \cup B \subset \cup_{k=1}^{ \infty} L_k =
    \left( \cup_{k=1}^{ \infty} I_k \right)
    \cup \left( \cup_{k=1}^{ \infty} J_k \right)
    $. 

    Then we have 
    \begin{gather*}
        \lvert A \cup B \rvert \leq  
        \left( \sum_{k=1}^{ \infty} l(L_k) \right) = 
        \left( \sum_{k=1}^{ \infty} l(I_k) + l(J_k) \right)
        =
        \left( \sum_{k=1}^{ \infty} l(I_k) \right)
        + \left( \sum_{k=1}^{ \infty} l(J_k) \right) 
    \end{gather*}
    By taking the infimum on the right side and using the fact that the 
    infimum of the sum of sets is the sum of infimums we get
    \begin{gather*}
        \lvert A \cup B \rvert \leq 0 + \lvert A \rvert = \lvert A \rvert.
    \end{gather*}
    Another inequality $ \lvert A \rvert \leq \lvert A \cup B \rvert  $  holds because 
    outer measure preserves order.
    Thus, $ \lvert A \rvert  = \lvert A \cup B \rvert  $.
\end{sol}
\end{problem}

% 2
\setcounter{problem}{1}
\begin{problem}
\begin{sol}
    If $ t = 0 $, then $ tA = \{ 0 \} $, therefore $ \lvert tA \rvert = 0 $, since
    it is finite. Also, $ \lvert t \rvert \lvert A \rvert = 0 $
    (even if $ \lvert A \rvert = \infty $, by convention).

    Now assume that $ t \ne 0 $.
    Let $ I_1, \ldots $ s.t.
    $ A \subset \cup_{k=1}^{ \infty} I_k $.
    Then $ tA \subset \cup_{k=1}^{ \infty} t I_k $.
    Easy to see, that $ \ell(t \cdot  I_k) = \lvert t \rvert  \cdot \ell(I_k) $. 
    So $ \lvert t A \rvert \leq \sum_{k=1}^{ \infty} \ell (t I_k) =
    \lvert t \rvert  \sum_{k=1}^{ \infty} \ell (I_k) $. 
    By taking infimum, we obtain
    $ \lvert t A \rvert \leq \lvert t \rvert \lvert A \rvert $.
    
    The opposite direction
    \begin{gather*}
        \lvert A \rvert  = \lvert \frac{1}{t} (tA) \rvert \leq
        \frac{1}{\lvert t \rvert} \lvert tA \rvert,
    \end{gather*}
    where we just used the first inequality with $ t  $ replaced with $ \frac{1}{t} $ and
    $ A $ replaced with $ tA $. Finally, we get
    $ \lvert t \rvert \lvert A \rvert \leq \lvert t A \rvert $.
\end{sol} 
\end{problem}

 
\end{document}
